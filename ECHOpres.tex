
\documentclass{beamer}
\usepackage[utf8]{inputenc}
\usetheme{Madrid} 
\setbeamertemplate{itemize items}[default]
\setbeamertemplate{enumerate items}[default]
\usepackage{graphicx}
\graphicspath{ {images/} }
\usepackage{multicol}
\usepackage{wrapfig}
\usepackage{amsmath, amsthm, amssymb, mathrsfs}
\usepackage{sansmathaccent}
\pdfmapfile{+sansmathaccent.map}

\usepackage[
    backend=biber,
    style=authoryear-icomp,
    sortlocale=de_DE,
    natbib=true,
    url=false, 
    doi=true,
    eprint=false
]{biblatex}
\addbibresource{/Users/Seth/Documents/BibTex/ICRAF_Literature-Methods.bib}
 
%Information to be included in the title page:
\title{Your Eyes in the Sky}
\subtitle{Using Remote Sensing to Improve Planning, Data Collection and Impact Evaluation}
\author{Seth Morgan}
\institute[UIUC] 
{
  \inst{1}%
  The Department of Agricultural \& Consumer Economics\\
  The University of Illinois at Urbana-Champaign\\
  The World Agroforestry Centre (ICRAF)
  
}

\date[ECHO 2016] % (optional)
{ECHO Best Practices in Highland Areas Symposium, November 2016}
 
 
 
\begin{document}
 
\frame{\titlepage}
 
\begin{frame}
\frametitle{Table of Contents}
\tableofcontents
\end{frame}

\section{Introduction}

\begin{frame}
\frametitle{Background}
\begin{columns}
 \begin{column}[T]{.6\textwidth}
  \includegraphics[width=\columnwidth]{visign}
 \end{column}
 \begin{column}[T]{.4\textwidth}
  \textbf{Vi Agroforestry:}
  \begin{itemize}
   \item Swedish NGO
   \item Reforestation and agroforestry promotion in Western Kenya since 1980s
   \item Expanded to Bungoma and Kakamega in 2008
   \item Site chosen for long-term impact evaluation
  \end{itemize}
 \end{column}
\end{columns}
\end{frame}

\begin{frame}
\frametitle{Background}
 \begin{columns}
  \begin{column}[T]{0.4\textwidth}
   \includegraphics[width=\columnwidth]{icrafspia}
  \end{column}
  \begin{column}[T]{0.6\textwidth}
   \textbf{ICRAF:}
   \begin{itemize}
    \item CGIAR Standing Panel on Impact Assessment grant for underevaluated areas
	\item ICRAF engaged in partnership with Vi through 90s and 2000s
	\item Extent of ICRAF knowledge transfer is empirical question
	\item Opportunity to examine long-term household-level impacts of agroforestry
   \end{itemize}
  \end{column}
 \end{columns}
\end{frame}

\begin{frame}
\frametitle{Research Questions}

\begin{block}{Primary Research Question}
What are the downstream socio-economic and land health effects of agroforestry adoption in Kenya?
\end{block}

\begin{block}{Sub-Question}
Can we improve the rigor of a pseudo-experimental research design by choosing a comparison group using matching over geospatial variables?
\end{block}

\end{frame}

\begin{frame}
\frametitle{Why Use Maps \& Remote Sensing?}

\begin{block}{Why We Needed Spatial Data}
 \begin{enumerate}
  \item No baseline data for non-program participants
  \item Treatment (the agroforestry extension program) assigned non-randomly
  \item Reason to believe geography and landscape play a part in treatment assignment and impacts
 \end{enumerate}
\end{block}
\pause

\begin{block}{Other Reasons to Look at Spatial Data}
\begin{itemize}
\item Space matters
\item People are visual
\item The available data is improving quickly
\item Available datasets include information on soil, rainfall, tree cover, population, elevation and more
\item Freely available tools
\end{itemize}
\end{block}
\end{frame}

\begin{frame}
\frametitle{What is Available?}

\begin{itemize}
	\item Geography
	\begin{itemize}
		\item Global Administrative Areas: gadm.org
		\item ILRI GIS database
		\item FAO Geonetwork: fao.org/geonetwork
		\item OpenStreetMap
		\item USGS Earth Explorer: earthexplorer.usgs.gov
		\item WorldPop Population Density: worldpop.org.uk
	\end{itemize}
	\item Forests and Tree Cover
	\begin{itemize}
		\item Global Land Cover Facility: landcover.org
		\item Global Forest Change: earthenginepartners.appspot.com
		\item USGS Land Cover: landcover.usgs.gov
	\end{itemize}
	\item Soil 
	\begin{itemize}
		\item ICRAF Landscapes Portal: landscapeportal.org
		\item Africa Soil Information System: africasoils.net
	\end{itemize}
	\item Climate
	\begin{itemize}
		\item CHIRPS Precipitation: chg.geog.ucsb.edu/data/chirps
		\item NOAA Climate Data Online: ncdc.noaa.gov/cdo-web
	\end{itemize}
\end{itemize}
	
\end{frame}

\begin{frame}
\frametitle{How can I analyse this stuff?}
\begin{columns}
 \begin{column}[T]{0.6\textwidth}
  \includegraphics[width=\columnwidth]{earthexplorer}
 \end{column}
 \begin{column}[T]{0.4\textwidth}
  \textbf{Three kinds of tools:}
	\begin{itemize}
	 \item Geographic Information System (GIS)
	   \begin{itemize}
		\item QGIS
		\item ArcGIS
		\item R
	   \end{itemize}
	 \item Data Collection Tool
	  \begin{itemize}
		\item ODK Collect
		\item ona.io
	  \end{itemize}
	 \item Data Analysis Tool
	  \begin{itemize}
		\item R
		\item Stata
		\item SPSS
		\item Excel
		\item LibreOffice Calc
	  \end{itemize}
	\end{itemize}
 \end{column}
\end{columns}
	
\end{frame}

\section{Planning: QGIS}

\begin{frame}
\frametitle{Intro to QGIS}
	\begin{columns}
	 \begin{column}[T]{.6\textwidth}
		\includegraphics[width=0.75\columnwidth]{qgislogo}
		
		\includegraphics[width=\columnwidth]{qgisscreenshot}
	 \end{column}
	 \begin{column}[T]{.4\textwidth}
		\begin{itemize}
		\item Free GIS system
		\item Access to plugins for OpenStreetMap, Google Maps
		\item Can utilize algorithms from R, GRASS & SAGA for geoprocessing
		\item Can build map projects and assemble map layouts for publication and sharing
		\end{itemize}
	 \end{column}	
	\end{columns}
\end{frame}

\begin{frame}
\frametitle{Types of Data}
\textbf{Vector Data}: points, lines and ploygons
 \begin{figure}[h]
  \centering
  \includegraphics[width=0.75\textwidth]{vectorexample}
 \end{figure}
 
 \begin{itemize}
 \item Used for depicting roads, borders and centers of things.
 \item \textbf{File Types}: .shp (plus .dbf & .shx), .rds
 \end{itemize}
\end{frame}

\begin{frame}
\frametitle{Types of Data}
\begin{columns}
 \begin{column}[T]{.6\textwidth}
   \includegraphics[width=\columnwidth]{rasterexample}
 \end{column}  
 \begin{column}[T]{0.4\textwidth}
  \textbf{Raster Data}: Grid cells with numbered values
  \begin{itemize}
  \item Used for terrain and representing the value of variables like rainfall across space.
  \item Can overlay with vectors and perform calculations on values
  \item \textbf{File Types}: .tif, .asc, .jpg, .png 
  \end{itemize}
 \end{column}
\end{columns}
\end{frame}

\begin{frame}
\frametitle{Examining Data}
 \textbf{Attribute Table}: Spreadsheet data connected to your shapefile
 \begin{figure}[h]
  \includegraphics[width=0.75\textwidth]{attributetable}
 \end{figure}
 
  \begin{itemize}
  \item Data is saved in .dbf file
  \item Limited manipulation within QGIS
  \item Can be analysed or edited in Excel or other program
  \end{itemize}

\end{frame}

\begin{frame}
\frametitle{Geoprocessing Functions}
 \begin{columns}
  \begin{column}[T]{0.6\textwidth}
   \includegraphics[width=\columnwidth]{geoprocessing}
  \end{column}
  
  \begin{column}[T]{0.4\textwidth}
   \textbf{Useful Functions:}
   \begin{itemize}
    \item Clip
    \item Buffer
    \item v.distance (GRASS)
    \item Zonal Statistics
    \item v.what.rast.points (GRASS) or Add raster values to points (SAGA)
   \end{itemize}
  \end{column}
 \end{columns}
\end{frame}

\begin{frame}
\frametitle{Map Layouts}
 \begin{columns}
  \begin{column}[T]{0.8\textwidth}
   \includegraphics[width=\columnwidth]{treecoverlayout}
  \end{column}
  \begin{column}[T]{0.2\textwidth}
   \textbf{TOSSLAD:}
	\begin{itemize}
	 \item Title
	 \item Origin
	 \item Source
	 \item Scale
	 \item Legend
	 \item Author
	 \item Date
	\end{itemize}
  \end{column}	
 \end{columns}
\end{frame}

\begin{frame}
\frametitle{Example: Site Scoping and Sublocation Matching}
\begin{columns}
 \begin{column}[T]{0.4\textwidth}
  \includegraphics[width=\columnwidth]{UyomaSoil}
 \end{column}
 \begin{column}[T]{0.6\textwidth}
  \includegraphics[width=\columnwidth]{SirisiaSubLoc}
 \end{column}
\end{columns}
  
\end{frame}

\section{Data Collection: ODK Collect and ona.io}

\begin{frame}
\frametitle{ODK Collect}
 \begin{columns}
  \begin{column}[T]{0.4\textwidth}
   \includegraphics[width=\columnwidth]{odkscreenlogo}
  \end{column}  
  \begin{column}[T]{0.6\textwidth}
   \begin{itemize}
    \item Mobile platform for data collection
	\item Takes geocodes, photos, text, multiple choice
	\item Works offline, then uploads with network
	\item Requires Android smartphone or table with SD card
   \end{itemize}
  \end{column}
 \end{columns}
\end{frame}

\begin{frame}
\frametitle{Creating Forms for ODK}
  \begin{columns}
  \begin{column}[T]{0.6\textwidth}
   \includegraphics[width=\columnwidth]{xlsformodkbuild}
  \end{column}  
  \begin{column}[T]{0.4\textwidth}
   \begin{itemize}
    \item build.opendatakit.org provides free form builder app
	\item For longer or more complex forms, create in Excel or LibreOffice
	\item Instructions available at xlsform.org
	\item Upload to ona.io or opendatakit.org/xiframe to convert for use in ODK Collect
   \end{itemize}
  \end{column}
 \end{columns}

\end{frame}

\begin{frame}
\frametitle{Aggregating data on ona.io}
 \begin{columns}
  \begin{column}[T]{0.6\textwidth}
   \includegraphics[width=\columnwidth]{onahome}
  \end{column}  
  \begin{column}[T]{0.4\textwidth}
  
   \begin{itemize}
    \item Free account for public data
	\item Collect and store data in the cloud
	\item Display geocoded data on a map
	\item Export survey data as a spreadsheet
   \end{itemize}
   
    \textbf{Alternatives:}
    \begin{itemize}
	 \item KoBo Collect
	 \item CommCare
	 \item ODK Briefcase 
	\end{itemize}
	
  \end{column}
 \end{columns}
\end{frame}

\begin{frame}
\frametitle{Example: Village Data Collection}
\begin{columns}
 \begin{column}[T]{0.6\textwidth}
  \includegraphics[width=\columnwidth]{mamapriscah}
 \end{column}
 \begin{column}[T]{0.4\textwidth}
  \textbf{Village Survey Questions}
  \begin{itemize}
   \item How many households in the village?
   \item How many active farmer groups?
   \item When were the groups formed?
   \item What activities do they pursue?
   \item Geocode collection
  \end{itemize}
 \end{column}
\end{columns}

\end{frame}


\section{Analysis: R}

\begin{frame}
\frametitle{What is R?}
\begin{columns}
 \begin{column}[T]{0.6\textwidth}
  \includegraphics[width=0.6\columnwidth]{rlogos}
  
  \includegraphics[width=\columnwidth]{rstudio}
 \end{column} 
 \begin{column}[T]{0.4\textwidth}
  "A language and environment for statistical computing and graphics"
  \medskip
  
  \textbf{Essential Packages:}
  \begin{itemize}
   \item rstudio 
   \item dplyr
   \item readr
   \item tidyr
   \item tidyverse
   \item ggplot2
  \end{itemize}
 \end{column}
\end{columns}

\end{frame}

\begin{frame}
\frametitle{R Pros and Cons}
 \begin{columns}
  \begin{column}[T]{0.5\textwidth}
   \textbf{Why Use R}
   \begin{itemize}
    \item Extremely adaptable
    \item Widely available
    \item Results are reproducible
    \item Enthusiastic user community
	\item Growing library of packages and documentation
   \end{itemize}
  \end{column} 
  \begin{column}[T]{0.5\textwidth}
   \textbf{Why Not to Use R}
   \begin{itemize}
    \item Not intuitive for beginners
	\item Minimal GUI: have to learn some code
    \item Difficult to examine data
    \item No centralized help system
   \end{itemize}
  \end{column}
 \end{columns}

\end{frame}

\begin{frame}
\frametitle{Example: Village Matching}

\textbf{Objective:} assemble a sampling frame made up of villages in which treatment and comparison villages are statistically indistinguishable across the following variables:

\begin{itemize}
\item Number of Households 
\item Average Soil Sand Content \parencite{Vagen2016}
\item Average Soil pH \parencite{Vagen2016}
\item Average Soil Organic Carbon \parencite{Vagen2016}
\item Average Tree Cover in 2005 \parencite{Sexton2013}
\item Elevation \parencite{Jarvis2008, Kruskac}
\item Average Population Density in 2010 and 2015 \parencite{Stevens2015}
\item Average Rainfall \parencite{Funk2015a}
\item Distance to Tarmac Road
\item Binary for Villages 0.25 m from Tarmac Road ("on road")
\item Binary for presence of microfinance activities
\end{itemize}

\end{frame}

\begin{frame}
\frametitle{Village Matching Model}
\textbf{Propensity Score:} The probability of receiving treatment, conditional on the covariates.

\begin{equation*}
e(X_i) = Pr(T_i = 1 | X_i)
\end{equation}

The score is generated with a probit regression, a regression model in which the outcome is binary (0/1, Treatment/Control):

\begin{equation*}
z = X\beta + \epsilon
\end{equation}

Where z is an unobserved variable and y is the observed binary corresponding to treatment assignment such that:

\begin{equation*}
y_i= \left \{ 
	\begin{tabular}{c}
	1 if z_i \geq 0 \\
	0 if z_i \leq 0 \\
	\end{tabular}
\end{equation} 

\end{frame}

\begin{frame}
\frametitle{Matching Results: Selected Village Points}
\includegraphics[width=\textwidth]{selectedvils}

\end{frame}

\begin{frame}
\frametitle{Matching Results: Selected Village Points}

\begin{table}[ht]\centering
\caption{Balance Statistics for Selected Villages}
\begin{tabular}{rrrr}
 \hline
 & Mean Difference & P-Value \\ 
  \hline
Households  &       16.83         &      (0.82)\\
Sand        &      -176.0         &     (-1.25)\\
pH          &      -1.950         &     (-1.14)\\
Tree Cover 2005      &      -0.134         &     (-0.30)\\
Elevation       &      -12.90         &     (-0.49)\\
Population 2010       &      0.0789         &      (0.36)\\
Soil Organic Carbon         &       27.20         &      (0.17)\\
Average Rainfall  &       3.614         &      (0.99)\\
Distance too Major Road  &   -0.000186         &     (-0.06)\\
On Major Road     &      0.0323         &      (0.58)\\
\hline
\end{tabular}
\end{table}

\end{frame}

\begin{frame}
\frametitle{Matching Results: Actual Village Points}
\includegraphics[width=\textwidth]{actualvils}

\end{frame}

\begin{frame}
\frametitle{Matching Results: Actual Village Points}

% latex table generated in R 3.3.1 by xtable 1.8-2 package
% Sun Oct 30 20:47:15 2016
\begin{table}[ht]
\centering
\caption{Balance Statistics Post Data Collection} 
\begin{tabular}{rrrr}
  \hline
 & Comparison & Treatment & P-Value \\ 
  \hline
pH & 5.94 & 5.96 & 0.19 \\ 
  Sand & 19.38 & 20.63 & 0.38 \\ 
  Soil Organic Carbon & 26.67 & 24.37 & 0.05 \\ 
  Elevation & 1568.06 & 1569.20 & 0.97 \\ 
  Population Density 2010 & 4.43 & 4.39 & 0.86 \\ 
  Population Density 2015 & 5.54 & 5.49 & 0.86 \\ 
  Average Rainfall & 138.76 & 134.51 & 0.15 \\ 
  Tree Cover 2005 & 6.15 & 5.99 & 0.73 \\ 
  Distance to Major Road & 0.03 & 0.03 & 0.75 \\ 
  On Major Road & 0.03 & 0.02 & 0.55 \\ 
   \hline
\end{tabular}

\end{table}

\end{frame}


\section{Conclusion}

\begin{frame}
\frametitle{Lessons Learned}

\textbf{Lessons for Impact Evaluations}
\begin{itemize}
 \item Scoping takes time
 \item Take the time to get all stakeholders on board
 \item Recruiting comparisons takes 2-3 times the investment as treatment
 \item Respect respondents' wishes and expectations, even when you can't fulfil them
 \item Remote sensing data is useful, but not perfect
 \item Use remote sensing to complement local information, not supplant it
 \item Take note of opportunities for capacity building with local staff
\end{itemize}

\end{frame}

\begin{frame}
\frametitle{Thanks}
\centering

The University of Illinois at Urbana Champaign Department of Agricultural and Consumer Economics
\medskip

The World Agroforestry Centre
\medskip

Vi Agroforestry Kenya
\medskip

Borlaug Fellowship in Food Security
\medskip

Dr. Katherine Baylis (UIUC)
\medskip

Dr. Karl Hughes (ICRAF)

\end{frame}


\begin{frame}
\frametitle{Questions and Follow-up}

\begin{block}{Contact}
smorgan9@illinois.edu

sethhenrymorgan@gmail.com
\end{block}

\end{frame}

\begin{frame}[shrink]
\frametitle{Bibliography}
 \printbibliography
\end{frame}

\end{document}

